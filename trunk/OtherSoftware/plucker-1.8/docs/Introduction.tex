%
% $Id: Introduction.tex,v 1.9 2003/07/31 15:12:06 desrod Exp $
%
\chapter{Preface}

This manual describes Plucker, an off-line HTML viewer for your Palm OS
hand-held device. Plucker increase the utility of your hand-held device
by letting you view web pages and any document that can be converted to
HTML or text. Plucker has many advanced features including the ability
to read web pages with embedded images, an advanced find function, the
ability to open an e-mail form when tapping on mail-links in web
documents, an impressive compression ratio for the documents and an open,
documented DB format. It can also be customized for your specific needs.
And not to be forgotten the source code is available for your perusal.

The general layout of this document is as follows. First we will list
the necessary components you need to run Plucker, and we will give you
a short introduction on how Plucker works. Then we will give detailed
instructions about the installation on your system if there are any
specific parts for the operating system you use. After the setup
routine of Plucker we will show you how to customize Plucker to get
the contents.

As a general guideline we strongly encourage you to read at least the
first three chapters of this handbook. When you understand how Plucker
works it is easy to set up the advanced features using the reference
part of this book.

\section{System Requirements}

\subsection{Device}

Plucker will run on Palm Pro, Palm III, Palm IIIx/IIIe,
Palm V, PalmVx, Handspring Visor, TRGPro and other Palm OS
hand-held devices utilizing PalmOS 2.0 and higher.\\

Plucker requires \viewerram{} of free space for the viewer application
and an additional \libram{} for the shared library if the ZLib
compression should be used (only PalmOS 3.0 and higher can use the
zlib compression).\\

Additional free space is required for your data and at run-time for
decompression. An alert will be display if there is not enough free
space on your device. Note that Plucker can use documents stored in
Flash memory and on read-only expansion cards.\\

\note{} The higher bit depth (more colors) you use for the images
the more memory they will require, so it is recommended not to use
higher bit depth than necessary. See below on how to influence the
bit depth Plucker uses.

\subsection{How Plucker works}

Let us start with a short description on how Plucker works. This will
help you to understand certain procedures necessary to use Plucker to
its best.

First of all let us mention that there are three parts:

\begin{enumerate}

  \item The Viewer: this is the actual application you run on your
  hand-held. It is used to display your documents as the name suggests.

  \item The Parser: more or less the brain of Plucker. The parser
  gathers contents from the Internet and converts it to a suitable
  format for the viewer. This part is installed on your workstation,
  not on your hand-held, so unlike other products Plucker uses your
  powerful desktop for the hard work.

  \item Some transfer tool which sends the pages you gathered using the
  parser to your hand-held. Actually, this is not part of Plucker
  itself and how this is done depends on the platform you using.
  For example, on Unix or OS/2 you might want to use Pilot-Link
  whereas on Windows you might prefer to use the HotSync Manager.
  It is only necessary that you understand that you must transfer the
  data in any way to your hand-held. We stress this here as many users
  who converted from some other application to Plucker missed this
  point.

\end{enumerate}

Consequences of this are immediate:

\begin{itemize}

  \item \textbf{Before} you are able to view any page from the Net on
  your hand-held you \textbf{have to} run the parser (sometime also 
  called \emph{hotsync} in this document). Normally you will invoke 
  it via some script, which we will explain later on.  Even on Windows 
  Plucker does \textbf{not} start the gathering process automatically 
  if you press the HotSync button on your cradle, as some other off-line 
  web readers do. We want to stress that you \textbf{must} start the 
  parser manually (or by some tool that does this automatically for you), 
  and after running the parser you \textbf{must} sync your device to get 
  the documents onto your hand-held. What sounds a bit difficult offers 
  you in fact many advantages.

  \item Since the parser is separated from the viewer and the transfer
  tool you will not have any problems if your HotSync-cradle shares the 
  COM-port with your modem.

  \item It is possible to run the parser automatically at some
  specified time.  That is you can have Plucker to collect your
  newspaper each morning at six o'clock without requiring your 
  attention and then HotSync the retrieved data to your hand-held
  before you leave home. In fact the whole process can be automatized
  using e.g.\ cron jobs,  by a (shareware) tool like \name{AutoSync} 
  from \textit{\htmladdnormallink{http://www.rgps.com}{http://www.rgps.com}}
  or some similar package.

\end{itemize}

\subsection{Converters}

Before you can read a document (HTML/text file) it has to be converted
to a Plucker format. For this we need some \emph{converters}. The
provided parser tool is the most essential, but to get support for
images as well you might want to read on as we need some 3rd party
applications for that as well. That is, if you lack images in your
gathered web pages most likely you are missing some of the necessary 
tools.

The provided parser tool currently runs on Linux, OS/2, Windows,
Solaris, AIX and FreeBSD\. It should be possible to get it running on
any platform with Python support.\\

\note{} On OS/2 you must install Plucker on a HPFS-partition. Plucker
will not run on OS/2 if you are using the FAT file system as Plucker
requires long filenames and OS/2 implements the traditional way of FAT
without the \emph{extensions} known from Windows.\\

If you know of other platforms where Plucker can be used successfully
we would appreciate if you send a message to \devaddress{} informing us
about it and if possible also include information about additional
requirements to get it running.

\subsubsection{Additional Software}

The following tools are not included within the distribution of
Plucker. You will find a list where to obtain these tools below.

\begin{itemize}
  \item \iname{Python}: a scripting language. Uses in the parser
  to retrieve contents from the web and put it into Plucker format
  for the Palm OS hand-held device. Compression of the documents is also
  handled by the Python parser.

  Python is {\bf required} on all supported platforms. 

  \item At least one of the following: \iname{Python Image Library (PIL)},
  \iname{ImageMagick} or \iname{NetPBM}. These tools are used for the image
  handling, but are not needed if you pluck all your data without images.

  \item Some transfer tool to transport the retrieved data to your
  hand-held. Most likely you have this already installed.

\end{itemize}

\subsubsection{Where to find the required software}

Where you find the tools:

  \begin{itemize}
    \item Python      : \textit{\htmladdnormallink{http://www.python.org}{http://www.python.org}}
    \item PIL         : \textit{\htmladdnormallink{http://www.pythonware.com/library.htm}{http://www.pythonware.com/library.htm}}
    \item ImageMagick : \textit{\htmladdnormallink{http://www.wizards.dupont.com/cristy/ImageMagick.html}{http://www.wizards.dupont.com/cristy/ImageMagick.html}}
    \item NetPBM      : \textit{\htmladdnormallink{ftp://ftp.x.org/contrib/utilities/}{ftp://ftp.x.org/contrib/utilities/}}
    \item OS/2 versions: Current ports can be found on our web page at
\textit{\htmladdnormallink{http://www.stellarcom.org/plucker/os2\_bins}{http://www.stellarcom.org/plucker/os2\_bins}}
  \end{itemize}

\section{Conventions}

For this document we use some common conventions.\\

\begin{tabular}{p{1.5cm}p{10cm}}

 \option{Bold} & is used for options to commands.\\\\

 \name{Italic} & is used to indicate files, directories, commands and
                 program names when they appear in the body of a paragraph.\\\\

 \code{Constant Width} & is used in examples to show the contents of
                       files or the output from commands, and to indicate
                       environment variables and keywords in the body of a
                       paragraph.\\\\

 \indata{Constant Bold} & is used in examples to show input that is typed
                      literally by the user.\\\\

 {[ ]} & surrounds optional elements in a description of program syntax.
           (The brackets themselves should never be typed.)\\\\
\end{tabular}


\section{We'd Like Some Feedback}

We have verified all information in this User's Guide, but you might
find that features have changed.  Please let us know about any errors
you find, as well as any suggestions for improvements, by sending an
e-mail to \devaddress


