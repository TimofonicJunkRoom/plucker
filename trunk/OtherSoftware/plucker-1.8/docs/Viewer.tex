%
% $Id: Viewer.tex,v 1.17 2004/03/21 14:07:00 nordstrom Exp $
%
\chapter{Viewer}

Using Plucker's client application (in this chapter refered to as
the \emph{viewer}) is in many ways just like using a normal web browser
(except for the fact that you are not connected to the Internet).


\section{Basic Operations}

This section describes how to start the viewer and navigate between
different pages. If you are new to Plucker, read this section to get
used to its basic operations. Later sections will describe how to customize
the viewer in many ways, handling of multiple documents, the find function,
and how to add bookmarks.

\subsection{Starting The Viewer}

Tap on the Plucker icon in the Applications screen to start the viewer. 
The first time the viewer is started it will open the document library
and display the available document in RAM and, if the device is
equipped with external card(s), any documents on the card(s). Subsequent
launches of the viewer will start where you left off.

If ZLib compression has been used when creating the document (see 
\~ref{sec:ParserOutput}) then the ZLib shared library must be installed 
or the records in the document cannot be uncompressed. DOC compressed 
documents don't require any additional library.

\subsection{The Main Screen}

The viewer will render the pages as close to the original as possible,
using everything from italic and bold fonts to embedded images. It 
also support horizontal rules, ordered and unordered lists, headers
(using different font sizes), left, right, and center text alignment,
and justified text.

\begin{figure} [!htb]
\centerline{\epsffile{MainScreen.eps}}
\caption{\label{fig:MainScreen}The Main Screen.}
\end{figure}

The toolbar can be moved from the top to the bottom of the screen or
removed completely using preference settings (see \~ref{sec:Prefs}).

The viewer has a 32 page history, so you can easily move backward and 
forward through recent visited pages just as you would with your desktop 
browser.\\

There are several different ways to move within the current page and
perform different actions. You can move using the scrollbar, the scroll
buttons and if the tap action (see \~ref{sec:TapAction}), gesture action
(see \~ref{sec:Gestures}), and button action (see \~ref{sec:HWButtons})
are activated also by pen taps on the screen, gestures in the graffiti area
and by pressing the hardware buttons, respectively. The fastest way to
move within the page is to use the location selector, though. Tap on the
location (the percentage in the toolbar) and a dropdown menu will appear
with 11 different values, Top, 10\%, 20\%,\ldots 90\%, and Bottom, 
so that you can choose the location you want to jump to. Tapping the 
autoscroll buttons will cause the text to scroll by at the specified
rate (see \~ref{sec:Autoscroll}.\\

Just like a web browser the viewer underlines links, a solid line for
not visited pages and a dotted line for already visited pages. Tap on
an underlined link to open the page it refers to (the link will be
highlighted when you tap on it). Image links are not underlined but
it should be obvious when an embedded image is a link and when it
is not. When in doubt just tap on the image, if it highlights it is
an image link\ldots\\ 

Embedded images will be scaled down if they are too big to fit within
the available width of the screen. If the alternative image options
are activated (see \~ref{sec:Parser}), then the embedded images will 
be a link to a larger version of itself. Tap on the image and the 
larger version will be shown. The viewer will try to change the screen 
depth to the depth required by the Tbmp document and if that depth is 
not supported it will display an alert. To see all parts of the 
\emph{larger-than-screen} image, tap on it and drag the pen in all 
directions.\\

When you tap on an e-mail tag the viewer will open a form so that 
you can compose and send a message. The form will show as much as 
possible of the To: Cc: and Subj: lines. If they don't fit on one 
line a little arrow will appear next to the line to allow you to 
see the rest of the line. The message will be sent through the 
built-in Mail application (i.e.\ you have to enable the mail 
synchronization using the Hotsync options (Windows) or some external 
tool (other platforms) to send the mail).\\

\begin{figure} [!htb]
\centerline{\epsffile{Mail.eps}}
\caption{\label{fig:Mail}Mail form.}
\end{figure}

Some pages may have external references (i.e.\ links to pages that 
were not included because they were either filtered out using the 
\name{exclusionlist.txt} file, excluded by \code{STAYONHOST}/
\code{STAYBELOW} or was exceeding the maximum link depth).\\ 

\begin{figure} [!htb]
\centerline{\hspace{1cm}\epsffile{CopyURL.eps}}
\caption{\label{fig:CopyURL}Copy URL.}
\end{figure}

When such a link is followed the viewer will display a page with the
URL for the external reference. You can tap \option{Copy URL} and the 
URL will be added to a Memo in the Memo database.\\

The Memo will be titled \code{Plucker URLs <date> <time>}. All URLs 
added during a single Plucker session will be in the same Memo. If 
you have a \emph{Plucker} category in your Memo database, the Memos 
will be placed there. Otherwise they will be in the \emph{Unfiled} 
category.\\

Read section \~ref{sec:Depth} if you want to learn more about link 
depths and section \~ref{sec:Exclusionlist} for more info about the 
filtering feature.\\

\begin{latexonly}
To get the URL for an existing page choose \option{Options\textbar Details} 
from the menu bar (or use the Graffiti shortcut /D). In the Details dialog 
you can also choose to not show images in a page and to reset its status 
(i.e.\ the link to this page will once again be a solid link indicating it
as unread).
\end{latexonly}
\begin{htmlonly}
To get the URL for an existing page choose \option{Options|Details} from
the menu bar (or use the Graffiti shortcut /D). In the Details dialog you
can also choose to not show images in a page and to reset its status (i.e.\
the link to this page will once again be a solid link indicating it as
unread).
\end{htmlonly}


\section{Customization}\label{sec:Config}

The viewer allows many different customizations to maximize the utility
of your reader.

\subsection{Font and layout}\label{sec:Font}

In the \option{Font and Layout} dialog you can set various layout options.
Depending on your device, you may have up to five built-in font packages, and
on most devices if you install an appropriate custom font package, you will be
able to load it in by choosing it from the \option{User-loaded font} drop down
list.  You can also set line-spacing, paragraph-spacing and rotation options.\\

User loaded Plucker font packages may be placed either in your device's RAM, or
in VFS storage, if available.  Font packages put in VFS storage must go in one
of the following directories:\\

\begin{itemize}
  \item /Palm/Programs/Plucker/Fonts
  \item /Palm/Ebooks/Fonts
  \item /Palm/Launcher/
\end{itemize}

Note that if you are using anti-aliased fonts, only those rotations selected
when generating the font will be supported.\\

You can download some pre-built Plucker font packages and tools to generate
your own from
\textit{\htmladdnormallink{http://palmfontconv.sf.net}
{http://palmfontconv.sf.net}}.\\

Finally, you can choose whether the font and layout settings are allowed to
vary between documents.  If you do not choose this, then the choices you make
will apply to all documents.

\subsection{Keyboard and graffiti}\label{sec:Keyboard}

The \option{Set keyboard/graffiti} dialog lets you assign keyboard actions or
graffiti entries to various actions such as:\\

\begin{description}
  \item[None] Do nothing.
  \item[Full Page Up] Scroll one page up.
  \item[Full Page Down] Scroll one page down.
  \item[Half Page Up] Scroll half a page up.
  \item[Half Page Down] Scroll half a page down.
  \item[Go Back] Go backward in page history.
  \item[Go Forward] Go forward in page history.
  \item[Go Home] Go to home page.
\end{description}

You can just type in or draw a graffiti entry and then select the action
from the drop-down list.  There is also a second drop-down list for
special keys like \code{Tab}.  Note that all the actions assigned work in the
document view.\\

Only some actions work in the library, and only when assigned to one of the
``Special keys''.\\

You should turn off \option{Graffiti gestures} (see \~ref{sec:Gestures}) if
you are using this feature.  If you are using Graffiti~2, actions assigned
to multi-stroke characters will likely not work.

\subsection{Preferences}\label{sec:Prefs}

\begin{figure} [!htb]
\centerline{\epsffile{Prefs.eps}}
\caption{\label{fig:Prefs}The Preferences.}
\end{figure}

In the Preference dialog you can set several different options to 
change the way the viewer works.\\

\begin{description}
  \item[Enable strike-through] By enabling this option an strike-through 
  will be added to all visited links making it easier to see what links 
  you have already visited.

  \item[Enable underline tags] By enabling this option all links will
  be indicated with a dotted line and the solid line will instead be
  used for the HTML underline tag. 

  \item[Manual update of document list] On some devices it can be a very slow
  process to update the document list and this option makes it possible to
  turn off the automatically update of the list and instead you have to update
  the list manually using a menu option in the Document Library \~ref{sec:DBMgr}.

  \note{} This will give the result that new documents will not turn up in
  the list and that documents removed using an external tool still will be
  included in the list. A manual update of the list will fix that, though.

  \item[Command Toolbar Button] On a device running Palm OS 3.5 or later
  you can set one of the buttons for the command toolbar. Has no effect 
  for other versions of Palm OS.

  \item[Screen depth] Depending on the Palm model you can set different
  default screen depths.

  When the screen depth is lower than the bit depth of images in a page 
  the viewer will display \mbox{\textbf{[img \emph{N}bpp]}}, where \emph{N}
  is the depth of the image. If the device can support that bit depth, 
  you can change the screen depth to see the image.

  \hint{} The pages scroll faster when the device is using its default 
  bit depth, so it is recommended to only use a different bit depth when 
  necessary.

  \item[Scrollbar] You can have either a left scrollbar, a right scrollbar 
  or no scrollbar.

  \item[Toolbar] You can have the toolbar at the top or bottom of the page,
  or no toolbar at all.

\end{description}

\subsection{Tap Actions}\label{sec:TapAction}

\begin{figure} [!htb]
\centerline{\epsffile{TapAction.eps}}
\caption{\label{fig:TapAction}Tap Action Configuration.}
\end{figure}

The Tap Action dialog enables you to define different actions for pen
taps on the screen. The first two modes divide the screen into 4 sections
and and each section can be given an action, e.g.\ one of the following,

\begin{description}
  \item[None] Do nothing.
  \item[Full Page Up] Scroll one page up.
  \item[Full Page Down] Scroll one page down.
  \item[Half Page Up] Scroll half a page up.
  \item[Half Page Down] Scroll half a page down.
  \item[Go Back] Go backward in page history.
  \item[Go Forward] Go forward in page history.
  \item[Go Home] Go to home page.
\end{description}

The third mode enables you to tap on the screen and drag the pen up or
down to scroll the page.

\subsection{Gesture Actions}\label{sec:Gestures}

\begin{figure} [!htb]
\centerline{\epsffile{Gestures.eps}}
\caption{\label{fig:Gestures}Gesture Action Configuration.}
\end{figure}

The Gesture Action dialog enables you to define different actions for
gestures in the graffiti area. There are five different gestures, stroke
right, stroke left, stroke up, stroke down, and a graffiti tap. Each
gesture can be given an action, e.g.\ one of the following,

\begin{description}
  \item[None] Do nothing.
  \item[Full Page Up] Scroll one page up.
  \item[Full Page Down] Scroll one page down.
  \item[Half Page Up] Scroll half a page up.
  \item[Half Page Down] Scroll half a page down.
  \item[Go Back] Go backward in page history.
  \item[Go Forward] Go forward in page history.
  \item[Go Home] Go to home page.
\end{description}

\subsection{Button Actions}\label{sec:HWButtons}

\begin{figure} [!htb]
\centerline{\epsffile{HWButtons.eps}}
\caption{\label{fig:HWButtons}Button Action Configuration.}
\end{figure}

The Button Action dialog enables you to define different actions for the
four hardware buttons, e.g.\ one of the following,

\begin{description}
  \item[None] Do nothing.
  \item[Full Page Up] Scroll one page up.
  \item[Full Page Down] Scroll one page down.
  \item[Half Page Up] Scroll half a page up.
  \item[Half Page Down] Scroll half a page down.
  \item[Go Back] Go backward in page history.
  \item[Go Forward] Go forward in page history.
  \item[Go Home] Go to home page.
\end{description}

\subsection{Autoscroll Options}\label{sec:Autoscroll}

\begin{figure} [!htb]
\centerline{\epsffile{Autoscroll.eps}}
\caption{\label{fig:Autoscroll}Autoscroll and Scroll Options.}
\end{figure}

The Autoscroll and Scroll Options enables you to define the amount that
scroll during each jump (in either pixels or pages). Autoscroll direction
can be specified as either up or down. The delay interval between each
jump can also be configured. You can prevent the device from automatically
shutting off during an autoscroll; autoscroll will then continue until the
end of the page is reached, the screen is tapped, or the power button is
clicked.

\subsection{Lookup Options}\label{sec:Lookup}

You can set a special word-tap action accessed by tapping on a word.
\begin{latexonly}
You can have this feature active all the time, so that whenever you
tap on a word other than one in a hyperlink the action is activated,
or you can activate it on demand for the next tap via
\option{View\textbar Lookup word} or by binding the \option{Lookup word}
action to a key, graffiti gesture, etc.
\end{latexonly}
\begin{htmlonly}
You can have this feature active all the time, so that whenever you
tap on a word other than one in a hyperlink the action is activated,
or you can activate it on demand for the next tap via
\option{View|Lookup word} or by binding the \option{Lookup word}
action to a key, graffiti gesture, etc.
\end{htmlonly}

To use the \option{Lookup in PPI} option, you need to have
the Plucker Plugin Interface (PPI) and a compatible dictionary
installed and configured (see \~ref{sec:PPI}).\\

\section{Find Function}\label{sec:Find}

\begin{latexonly}
The viewer uses its own find function to search through the
records in a document. Tap on the left lens in the toolbar,
choose \option{Go\textbar Find\ldots} or use the Graffiti shortcut /F
to start a new search.
\end{latexonly}
\begin{htmlonly}
The viewer uses its own find function to search through the
records in a document. Tap on the left lens in the toolbar,
choose \option{Go|Find\ldots} or use the Graffiti shortcut /F
to start a new search.
\end{htmlonly}
The last search pattern will be shown in the find field and
by using the search history you can recall any of the last
10 search patterns.\\

\begin{figure} [!htb]
\centerline{\epsffile{Find.eps}}
\caption{\label{fig:Find}The Find Function.}
\end{figure}

It is possible to make a case-insensitive or case-sensitive search and also to
choose between only searching in the current record, in all records in the
document, in all records starting with the present point, or recursively in sub-
pages, i.e., pages linked from this one and so on.  Moreover, you can choose to 
search for a phrase (all the characters entered must be in the document as
entered) or for a number of words in a paragraph in which case each space
separated sequence of characters in the search string must occur in the same
paragraph (but need not appear as a complete word).\\

More sophisticated than case-insensitive searches are transliteration
searches.  These transliterate the text in the document according to a
a fixed transliteration table before searching.  This can be used to search
for accented characters without knowing what the accentuation is or to search
for characters in the document's encoding that are difficult to enter via
Graffiti or that your Palm localization does not support, but that are
supported via a custom font.  You can make your own transliteration tables
via the tools in the source code \code{viewer/xlit} directory.  Each
transliteration table is a separate \code{.pdb} database.  Here are some
common ones.\\

\begin{description}
  \item[\name{Latin1.pdb}] Drop Latin 1 accents before searching.  Also, change
        Drop Latin 1 accentsall extended characters to ASCII characters.
        Searching for \code{e} will find \code{\'e}, \code{\`e}, \code{\"e} and
        \code{\^e}, etc.  If case the search is case insensitive, it will also
        find \code{\'E}, \code{\`E}, \code{\" E} and \code{\^E}.
  \item[\name{Latin2.pdb}] Drop Latin 2 accents before searching.  This is the
        same as \name{Latin1.pdb} but for the \code{ISO-8859-2}
        (Eastern European) encoding.
  \item[\name{CP1250.pdb}] Drop CP-1250 accents.  This is the same as
        \name{Latin2.pdb} but for the Windows Eastern European character
        encoding.
  \item[\name{NonASCIIToPeriod.pdb}] Non-ASCII to period.  All characters not
        in the ASCII printable range 32-126 are transliterated to a period
        (\code{.}).  Thus, you can search for \code{\`eglise} by entering
        \code{.glise} and for ``\code{goat}'' \emph{including} the curly quotes
        by entering \code{.goat.}.
  \item[\name{PBGreekKeys.pdb}] PB+GreekKeys to ASCII.  This transliterates
        GreekKeys encoded text (including PalmBible+ extensions) to ASCII
        using the same Greek transliteration as in the Online Bible and
        PalmBible+ programs.
\end{description}

Except for \name{PBGreekKeys.pdb}, all the transliterations are
\emph{symmetric}, i.e. the transliteration is applied not just to the document
before searching but also to the entered search text.  So if you use the
``Drop Latin 1 accents before searching'' option and search for \code{\'ab\^a},
you will also find \code{\`ab\"a}, \code{aba}, \code{ab\'a}, etc.\\

\begin{figure} [!htb]
\centerline{\epsffile{Result.eps}}
\caption{\label{fig:Result}The Result Page.}
\end{figure}

When searching in the current record the viewer will jump to the
location of a match and highlight it. If the search is for all
records in the document a result page will be shown with matches.
If you tap on one of the matches in the list the viewer will open
that record and just as when you search in the current record
it will jump to the location of the match and highlight it.\\

\begin{latexonly}
The right lens (and \option{Go\textbar Find Again\ldots} or  Graffiti
shortcut /A) will search for the next occurrence of a found word or
phrase with a single tap.
\end{latexonly}
\begin{htmlonly}
The right lens (and \option{Go|Find Again\ldots} or  Graffiti shortcut
/A) will search for the next occurrence of a found word or phrase with
a single tap.
\end{htmlonly}


\section{Document Library}\label{sec:DBMgr}

To store all data in only one PalmOS database wouldn't be very
flexible, so the viewer has support for multiple documents. By
using the viewer's Document Library you can switch between all
the documents you have loaded on your device. To the right of
each document are the creation date and the size (can be turned
off using menu options or preference settings). If there are
more documents than can fit in the list a scrollbar will appear.\\

\begin{figure} [!htb]
\centerline{\epsffile{DBMgr.eps}}
\caption{\label{fig:DBMgr}The Document Library.}
\end{figure}

When you tap on the document icon in front of each document
a pop-up list will appear providing you with different
operations for the selected document:

\begin{itemize}
  \item Open document
  \item Categorize document
  \item Delete document
  \item Rename document
  \item Beam document
\end{itemize}

Also, a tap on a document in the list will open it right away.\\

You can also select documents with jog-dial, five-way navigator,
or, if the list is sorted by document name, by entering the first couple
letters of the document name.\\

If the document is in flash (or read-only for some other reason)
it is not possible to delete or rename it. The viewer will show
an alert if you still try to perform such an action on the document.\\

The documents can be compressed either using the DOC method or
ZLib. Support for the DOC method is included by default in the
viewer so documents compressed with that method can always be
viewed. ZLib compressed documents requires the ZLib shared
library and if this library is not installed the viewer will
alert the user about this. Palm OS 2.x devices can only handle
DOC compressed documents.\\

Each document can be in one or several categories and it is possible
to filter the list of documents by selecting what categories that
should be shown.\\

The viewer will check the following directories on external cards
for Plucker documents,\\

\begin{itemize}
  \item /Palm/Programs/Plucker/
  \item /Palm/Ebooks/
  \item /Palm/Launcher/
\end{itemize}

On devices that can use more than one card the cards must be given
unique names.\\

\begin{figure} [!htb]
\centerline{\epsffile{Category.eps}}
\caption{\label{fig:Category}The Category Screen.}
\end{figure}

In the category screen you can add, remove and rename categories.
Tap on the pushbutton with the category name to toggle between hiding
and showing the category. Check the box MULTIPLE SELECT to allow more
than one category to be displayed at a time. To add a new category,
tap on an empty pushbutton. The document icon next to each category
provides you with the options to delete and rename a category. If you 
delete a category name, the documents filed under that name will become 
unfiled documents. To merge documents in different categories, rename 
one category to the other category name.

\section{Bookmarks}\label{sec:Bookmark}

Bookmarks are a fast way to move to user-defined positions in the 
documents.\\

\begin{figure} [!htb]
\centerline{\epsffile{Bookmarks.eps}}
\caption{\label{fig:Bookmarks}Bookmarks.}
\end{figure}

Edit Bookmark will display a list of all defined bookmarks and 
allow you to delete bookmarks. To use a bookmark, just tap on the 
Bookmark icon and choose the bookmark you want to jump to.

