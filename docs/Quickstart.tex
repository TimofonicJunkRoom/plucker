%
% $Id: Quickstart.tex,v 1.17 2003/07/31 15:12:06 desrod Exp $
%
\chapter{Installation and Quickstart Manual}

This chapter describes how to set up Plucker on your system and how to
configure it to gather contents for you. It does not cover all
features of Plucker for which you might refer to later chapters of
this handbook, but it will give you an introduction on how to use
Plucker's basic features. Many of the advanced topics are not covered
here to make this introduction readable even for beginners and as
simple as possible. Once you got the generals procedures of Plucker it
will be easy for you to understand the later parts of this handbook
which are meant mainly as reference guide.

\section{OS specific instructions}

First of all we will list some of the OS specific installation
procedures as well as OS specific things you should know to run
Plucker.

\subsection{OS/2}

Basically there are two ways to install Plucker on OS/2:

\begin{itemize}
  \item The Expert: You may obtain all packages you need to run
  Plucker separately and set them up by hand. This would be the
  preferred way if you have already several packages installed on your
  system (unless you work a lot with Unix-style tools on OS/2 this is
  not very likely the case) and if you have some experiences setting
  up such stuff.

  \item The Easy way: Just get the WarpIn-Installation packages and
  run install. This is the preferred way for new and/or inexperienced
  users and if you do not have the main parts to run Plucker installed
  on your system. \textbf{Unless specific circumstances or
  environments we encourage you to use this way!} The OS/2-installer
  will do no harm to your system and it will make it quite easy to
  de-install Plucker.
\end{itemize}

In any way you will get all files needed on our OS/2 page at

\textit{\htmladdnormallink{http://www.stellarcom.org/plucker/os2\_bins/index.html}{http://www.stellarcom.org/plucker/os2_bins/index.html}}.

On this page you will find all tools necessary to set up Plucker
either way you choose.\\

\note You need to install Plucker on a HPFS-partition. It will not
work on FAT file systems.\\

\note Currently there persist still some problems with image
conversion on OS/2 so you might not be able to use images in your
documents. We work on a solution to that issue.


After setting up Plucker please refer to chapter \~ref{sec:Parser} on
how to set up the necessary configuration files for Plucker.

\subsubsection{Using the Installer}

As stated above this is very easy indeed. Follow these steps:

\begin{enumerate}
  \item After downloading \name{PluckerInstaller.zip} just unzip it
  using your favorite tool to handle ZIP-files. (e.g.\ just type
  \code{unzip PluckerInstaller.zip} from your shell.)

  \item Copy the \name{WarpIn} folder to whatever location you want to
  store \name{WarpIn} at. If you have WarpIn already installed you
  may skip this step. (But ensure that you have the same or a newer
  version of WarpIn.) After this open that folder and double-click on
  \name{\name{WarpIn}.exe}. This will create the necessary program
  objects and associations to use \name{WarpIn}. Note that
  \name{WarpIn} is not written explicitly for Plucker and that we only
  use it. For details about \name{WarpIn} refer to its \name{README}
  files and documentation.  \name{WarpIn} is Freeware.\\

\note Once you have installed \name{WarpIn} on your system you can use
it to install any other product that is delivered in WPI format. An
increasing number of tools use this for there installation. 

  \item Now you are ready to install Plucker itself. This is quite
  easy. Open the folder where you unziped Plucker on the WPS and just
  double-click on the several WPI-packages (you may note that they are
  now associated with \name{WarpIn}.) This fires up \name{WarpIn}
  which asks you some basic questions on where you want to store the
  files and stuff like that. Simply follow the on-screen-instructions.

  \item The installer will create a folder on your Desktop called
  \name{Plucker for OS/2} which will contain all necessary objects for
  using Plucker. Additionally it will set up Python and Pilot-Link as
  well as ImageMagick for image conversion.
\end{enumerate}

That's it! Now you are ready to run Plucker on OS/2. \\

But before you start your first plucking we want to remind you to
read section \~ref{sec:basic} to set up Plucker to gather the right
contents.  You need to follow the steps there to get useful
results.\\

For detailed instructions how to use Plucker refer to the other
chapters in this documentation. You may find some hints on how to make
most out of Plucker in the appropriate locations within normal
documentation, just in case there is anything different on OS/2 than
on all the other platforms. You may note that this is very rarely the
case at all.\\

\note If you have any of the packages already installed that are
delivered by the installer (most likely this will be the case with
Pilot-Link) then simply skip it in the installation process for
Plucker.


\subsubsection{Setting up by hand}

This section will be quite short as we assume that you know what you
are doing. You will find all tools necessary for at
\textit{\htmladdnormallink{http://www.stellarcom.org/plucker/os2\_bins}{http://www.stellarcom.org/plucker/os2\_bins}}

The basic steps:

\begin{enumerate}
  \item Set up Python for OS/2

  \item In case you want to use zLIB-compression (see below) you need
  to install the additional zLIB.DLL for Python. (See our web site.)

  \item Install \name{ImageMagick} or \name{NetPBM} (whatever you
  prefer) for image conversion. Note that Plucker needs two additional
  converters for creating usual TBMP-format used on your hand-held. So
  we suggest that you get either of this packages from our web site.
  The necessary converters in current versions are included in the
  packages provided there.

  \item Install \name{Pilot-Link for OS/2}

  \item Unzip Plucker to the directory where you want to store the
  program

  \item Make sure that the environment variable \name{\$HOME} is set

  \item Create a directory called \code{.plucker} within your
  \name{\$HOME}-directory.

\end{enumerate}

If all packages are properly set up (this includes necessary changes
to \textit{CONFIG.SYS}, path-/libpath-entries and some environment
variables, refer to the docs of each package for this) you are ready
to run Plucker. Simply continue with the usual configuration.\\

Again: If you do not know what we where talking about here we suggest
that you use the installer. It will take care about all the necessary
settings. If you have any of the packages already installed that are
delivered by the installer  then simply skip it in the installation
process for Plucker.

\subsection{Unix/Linux}

There are several ways to get Plucker installed

\begin{itemize}
\item RPM package
\item DEB package
\item TAR package with binaries
\item TAR package with source code
\end{itemize}

If you install the RPM package, you only have to run
\name{plucker-setup} to set up the Plucker environment in you home
directory (see chapter \~ref{sec:Parser} for more info about changing
the description, configuration and exclusion list files)

If you have downloaded the binary TAR package, then do the following:

\begin{enumerate}
\item Unpack the file.
\item Go to the \name{unix} directory.
\item Run \name{./install-plucker}
\item Answer the questions about where the files should be installed.
\end{enumerate}

If you have downloaded the source package, you will do the same thing as
explained above, but this time you must have \textbf{all} development 
tools installed to be able to build all parts of Plucker.

\subsection{Windows}

On Windows we encurage you to use the graphical installer we provide.
It will take care about all steps necessary to set up Plucker
correctly. The follwoing steps are necessary to get Plucker running on
Windows using this installer:

\begin{enumerate}
  \item Download the Installerpackage for Windows from our web page and
  run it.

  \item Follow the on-screen instructions.
\end{enumerate}

\section{Basic customization - Or: Before you start} \label{sec:basic}

Now you have set up Plucker on your workstation. What is left to do?
\begin{itemize}
	\item Tell Plucker what pages to gather
	\item Customize the hotsync-script (only OS/2 and Windows)
	\item Start your first plucking and check that everything work as
	expected
\end{itemize}

The first goal is achived by creating a document called \emph{description
file} or \emph{home document}. The second is only necessary if you are 
running OS/2 or Windows and is normally done by editing a script called 
\emph{hotsync} (although you can name it whatever you like). And the last 
step is to run this script or to run \emph{plucker-build} if you are 
using Linux/Unix.

\subsection{Specifying the pages you want to view - The Home Document}

The home page is the first page you see when you start Plucker on your
PDA, and is also the page you see when you tap on the \emph{home} icon
(the little house) in the viewer on the Palm OS hand-held device. The
home page is by default created using the description file at
\name{\$HOME/.plucker/home.html}.\\

When you installed Plucker, a default description file was put in
\name{\$HOME/.plucker}. You can change this file in any text editor
and it doesn't require deep knowlege of HTML to create it. We will 
explain how to do this step by step in a minute.\\

Except for the \longcode{description file}{MAXDEPTH} and similar
attributes the \name{description file} is like any other HTML file.
This also means that you can view \name{home.html} in your normal
web browser (e.g. Netscape). In fact you can even use a normal
web page at some web server as your home document. See chapter
\~ref{sec:Parser} for details on how to do this.\\

Prior to performing a HotSync, you have to tell Plucker about where to
grab the pages that you want to view. Plucker starts by scanning the
\name{description file} (also refered to as \name{home document}) that
you have defined. As stated above if you did not define otherwise this
will be \name{\$HOME/.plucker/home.html}. \\

The parser finds any links in that file, and follows them. Each link
(e.g.\ something like \verb|<A HREF="...">|) is read from the Internet,
stored and parsed on your hard disk and included in the document that
you will later sync to your Palm. Let us explore the \name{description 
file} in more detail.\\

A simple, typical home document will look like this (without the
linenumbers, they are only added for easier reference later on):\\

\begin{verbatim}
[01]\t<H1>Plucker Home</H1>
[02]
[03]\t<H2>Plucker Information</H2>
[04]\t<P><A HREF="http://www.plkr.org"> Plucker home page</A><P>
[05]
[06]\t<H2>Linux links</H2>
[07]\t<A HREF="http://slashdot.org/index.pl?light=1&noboxes=1" NOIMAGES MAXDEPTH=1>Slashdot.org</A><P>
[08]\t<H2>News</H2>
[09]\t<A HREF="http://channel.nytimes.com/partners/palm-pilot/summ.html" MAXDEPTH=2>New York Times</A><P>
[10]\t<A HREF="http://www.news.com/Newsfeed/Avantgo/index.html" MAXDEPTH=2>C-Net NEWS.COM</A><P>
\end{verbatim}

Here you see several typical examples. First of all you may note that
this document follows the general outlines of normal HTML as stated
above. If you do not know HTML already, do not worry. What you need
here is really very easy. Let us look it trough line by line. (If you
already know HTML this will be somewhat boring to you.)\\

First you note that commands in HTML are enclosed in angle brackets.
Each command has a begin and an end always using the same tag, the
end marked by an additional slash in front of the tag's name. The first
and 3rd row e.g.\ create Headlines. The numbers simply specify different 
fontsizes.\\

\begin{latexonly}
More interesting is the 4th line. First it starts a new paragraph
(\textless P\textgreater) and then you see the most important tag in 
HTML: a link. Links have the following form:
\end{latexonly}
\begin{htmlonly}
More interesting is the 4th line. First it starts a new paragraph
(<P>) and then you see the most important tag in HTML: a link. Links 
have the following form:
\end{htmlonly}\\

\begin{verbatim}
<A HREF="http://www.plkr.org">Plucker home page</A>
\end{verbatim}

Enclosed in quotes you find the page they refer to (the URL) then
after an closing angle bracket you see the title of the link that
should be displayed to the user. You will see exactly this text within
Plucker on your home document.\\

Now you have the basic procedure how to tell Plucker to get a specific
web page for you. What does Plucker do if it finds a tag like this in
\name{home.html}? It will simply follow the URL you specified and get
this page. Note that as this page is plain HTML you can also view it
within your normal web browser and use the links the same way as
Plucker does it. That way you can easily check if all links are
working as you expected without the need to run the parser each time
and sync.\\

Well, it is nice to grab a web page, but what to do with a newsticker
that lists only the headlines? You want to retrieve the articles as 
well. Let's have a look at line 9. You will note that this
time the link is enhanced by an additional tag: \code{MAXDPTH=2}. This
is the way to instruct Plucker to retrieve deeper levels from a
web server. You can give \code{MAXDEPTH} any number as parameter.
\code{MAXDEPTH=2} means to load the target (linked-to) page, and any
targets within that page. This will just do the job. It will first
grab the headlines and then follow all linked pages to get the articles
as well.\\

\code{MAXDEPTH=3} will load the target page, its linked pages, and any
pages linked within there and so on. You really do not want to set
\code{MAXDEPTH} too high.  That could be very bad. A \code{MAXDEPTH}
of well under 50 would probably load the entire Internet, so you may
run into some storage problems\ldots\\

\code{MAXDEPTH} is one of the most important tags used to customize
the information to download. Another important tag is
\longcode{description file}{NOIMAGES}. A sample on how to use it can
be found in line 7. If you do not want to download any images you
should specify this tag.  Simply add it after the URL to
pluck. As you see in line 7 you can combine the various tags. That is
line 7 instructs Plucker to download only the title pages without
images. You see you can even explicitly specify a value of 1 to the
\code{MAXDEPTH} argument. This does not make much sense at the first
glance as our first example showed that leaving out the
\code{MAXDEPTH} statement will give the same result. But stop. The
reason to give a explicit \code{MAXDEPTH} is that you can define which
depth is the default for Plucker within Pluckers configuration file we
will talk about later on. So if you set the default depth to 2 e.g.\
but this page should definitely be plucked only to a depth of 1 you can
specify it explicitly here instead of defining a depth of 2 for all
other pages. (It is not a bad idea to define the depth explicitly for
all pages as one can see at first glance how deep Plucker will
work.)\\

Now in line 7 you see another possibility of Plucker. You might have
wondered about these funny chars that appear in the url. Well with
that definition you instruct Plucker to request the result of a so
called CGI and to pass some paraeters as well. A CGI is basically a
script run by a web provider to gather explicit informations e.g.\ from
a database. You do not need to worry about the details, the easiest
way to get the correct URL is always to point your web browser to the
page where the information is located and copy the URL from its
URL-field into the home-document of Plucker. Here we just wanted to
show that Plucker can even handle such funny URLs.\\

\hint{} If you specify a CGI for a newsticker and you always get the
same news it is most likely that the date of the issue is passed to
the script via the URL.\\

\note \code{MAXDEPTH} will most likely be sufficient for web pages that
where written for PDAs. If you are plucking \emph{normal} web pages be
careful with \code{MAXDEPTH}, as many pages contain a menu to
navigate the site and Plucker will follow these links as well. That is
Plucker does not distinguish between a menu or a \emph{normal} link.
That way a \code{MAXDEPTH=2} which is meant to download an overview page 
and the articles (e.g.\ of your newspaper) could easily result in Plucker
trying to retrieve the news archive of your favourite newspaper as well
(since it is linked from within the menu). So it is wise to attend 
the first runs of Plucker if you add new pages to your description file, 
especially if you do not use PDA-optimized pages. There are very 
effective ways to prevent Plucker from running into this kind of problems.  
Besides the \code{MAXDEPTH} and \code{NOIMAGES} tags there are various 
other tags that influlence the gathering process.  E.g.\ you can have 
Plucker to exclude specific links etc. See below in chapter 
\~ref{sec:Parser} for details about. We will not go into to much details 
here.\\

\hint{} Web pages created for handhelds have some advantages over normal
web pages. Plucker can retrieve normal web pages, no matter, but specially
designed pages are normally smaller, contain less graphics etc.\\

\hint{} Plucker can not handle \emph{frames}. Usually, pages designed
for handhelds do not contain frames\ldots\\

\hint{} A good collection of links to handheld friendly sites are
included with Plucker.\\

Our German speaking users can find sites with mostly German content at
\htmladdnormallink{http://www.palmtop-portal.de}{http://www.palmtop-portal.de}.

\section{Running the Parser}

Now that you have created your description file you are ready to run
the parser to get the pages. For our simple case here this does not
require much. Simply open a command line window, change to the
directory where you installed the python parser and run the following 
(\% is the shell prompt)\\

\indata{\% python PyPlucker/Spider.py -f PluckerDB}

(This call will work on any system, on Unix you can leave out the
explicit call to Python of course.) Make sure you have established a
connection to the Internet before you issue this call.\\

Now the parser will read your freshly created \name{home.html} follow
the links the way you defined them and finally write a file to your
harddisk called \name{PluckerDB.pdb}. In case of problems you might
want to add the additional parameter \code{-V2} to see where the error
occurs.\\

\note{} The spider has plenty of parameters. You will find a detailed
description of all of them in chapter \~ref{sec:Parser}. Additionally
to the parameters specified to the parser call you can set default
values in a file calles \name{plucker.ini} (on OS/2 and Windows) or
\name{.pluckerrc} (on Unix). So if something happens that you did not
expect it is a good idea to check this file. A detailed description of
the parameters for this file and its format can also be found in
chapter \~ref{sec:Parser} and within the file itself.

\section{Transfer to your Palm}

After you ran the parser on your system it will generate a document
for you, that is a file suitable for the viewer. Of course you need
to transfer this document to your handheld. By default we
suggest the usage of \code{pilot-xfer}. \code{pilot-xfer} is one of
the tools (more or less the central tool) of the famous
\name{pilot-link} package which is available for numerous platforms.
For Unix and OS/2 you might find both the binary and the source
distribution at \\

\htmladdnormallink{ftp://ryeham.ee.ryerson.ca/pub/PalmOS/}{ftp://ryeham.ee.ryerson.ca/pub/PalmOS/}\\

for Windows please have a look at Thilo Christs port at \\

\htmladdnormallink{http://linux.stud.fh-heilbronn.de/~christ/pilot-xfer}{http://linux.stud.fh-heilbronn.de/~christ/pilot-xfer}\\

Besides that Plucker offers its own conduit for Windows which
cooperates directly with the usual HotSync, and there is as well a
perl-based conduit available which you can still use to transfer the
data (though you might only want to use it in very special
circumstances, and in fact we discourage its usage).

\subsection{Pilot-Link}

Though the general usage of Pilot-Link package is beyond the scope of
this manual we will tell you enough to transfer your documents on the
Palm using Pilot-Link. On Unix and OS/2 you have mainly two choices:

\begin{enumerate}
  \item Create a document with the parser and transfer that one
  \item Create a so call \textit{cache-directory} and use the
        perl-based conduit.
\end{enumerate}

We encourage you to use the first method as it is easier and does not
require working perl and pilot-link's perl-interface on your system.
The perl based conduit may offer you some advantages though, so its
covered in the advanced topics. Please have a look there.\\

The usage of \name{pilot-xfer} is quite simple. Provided you set up
the package correctly (on OS/2 this is done by the installer) all
you need to do is

\indata{\% pilot-xfer -i FILE \ldots}

This will prompt you to press the hotsync button and then transfer the
document to your handheld. On OS/2 you can of course use the WPS to do
this. Simply open the folder where you stored your documents and
double-click on the one you want to install. (The GUI-installer has
set up he appropriate connections for you already.)\\

\hint{} In case you have more than one document to transfer you can
also use the \option{-r} switch and invoke \name{pilot-link} the
following way to transfer all documents within a directory at once

\indata{\% pilot-xfer -r DIR}
\\

\hint{} For OS/2 users. If you need to do this very often you might 
want to enhance the context menu of your document folder by this function.
Actually this is quite easy. Just create a program object that
contains the necessary call with all parameters needed (in the program
field just enter \code{pilot-xfer.exe}, in the parameters field you
might want to enter \code{-r \.}. Note the full-stop in the argument),
then open your folder settings and go to the \name{Menu} page. Now
simply drop the newly created program object in the lower list box and
name it appropriately using the \name{Settings..} button. Now the
context menu of your folder will have an additional entry to transfer
the whole folder to your hand-held. You might want to generate a shadow
of that folder on your desktop for easy access.

\subsection{The Windows conduit}

The Windows conduit is, unlike \name{pilot-link}, entirely written for
Plucker. For that it offers some parameters quite specific which we
will elaborate a bit about here. Additionally it transfers the data
not directly to the Palm but notifies the Hotsync Manager about a
document waiting for transfer, that is the document is sent to
your hand-held on your next HotSync-operation, not like on Unix when
you call the conduit.\\

The first difference in the usage is that it is thought to be used with
\textit{cache-directories}. That is it assumes that you run the parser
using \option{-c}. The new version supports the transfer of a complete
document as well though. \\

All parameters are in the form \option{-parameter=value}. Note that 
they do not follow POSIX conventions nor the normal DOS-conventions. 
Use one minus sign to separate all parameters. If you specify no 
parameter at all you will get a short help message.

The possible parameters are:

\begin{itemize}
  \item \option{-dir}: The path to \code{\$PLUCKERDIR}
  \item \option{-user}: Hotsync manager needs to know for which user the
document in question should be installed. (As you know the Hotsync
Manager allows for multiuser environments.) Use this switch to set
the username.

\note You must set this parameter. Without it no data will be
transfered.

  \item \option{-cache}: The name of the cache-directory if you use the
parser with the \option{-c} switch. The default value is \name{Cache}
if you do not provide that parameter. This path is given relative to
the path you provided with \option{-dir}.

  \item \option{-id}: Creator ID for the document. Default is \textbf{Plkr}. You
may not want to change this value.

  \item \option{-dbversion}: Versions number for the document.
  Default is 7. You may not want to change this value.
  \item \option{-dbname}: Name of the document, that is the name with
which the document will show up on the Palm. Default is \emph{PluckerDB}.
You will want to change this, and you have to change it if you want
multiple documents.
   \item \option{-fname}: Filename for the document on your local
harddisk. Default \emph{PluckerDB.pdb}. You need to change this for
multiple documents.
  \item \option{-fix}: 0x0D0A Sequences are converted to 0x0A. Normally
you should not need this parameter anymore.
  \item \option{-install}: Install a document. This parameter is needed
instead of \option{-cache} if you run the parser with the
\option{-f}-switch. \option{-user} must be set, \option{-dir} must not

\item \option{-verbose}: Set the verbosity Level. Default are
  \begin{itemize}
   \item[0] = Verbosity level 0 is silent except for errors
   (\emph{No news is good news})
   \item[1] = Verbosity level 1 gives progress status
   \item[2] = Verbosity level 2 is used for debugging
  \end{itemize}
  \item \option{-copyprev}: Set the CopyPrevention flag for the
        document. You may not be able to copy such documents from
        one Palm to another (easily). Useful for confidential data
        (but do not consider it to be a sufficient security lock.)
  \item \option{-backup}: Set the backup flag for the document,
        documents without this flag will not be backed up by hotsync.
  \item \option{-launchable}: Set the launchable flag for the document
Set this to see your document like a program in the Palm's launcher.
\end{itemize}
\note The last three parameters are only useful if you run the parser
with the \option{-c} parameter that is you let it write cache
directories instead of whole documents. Otherwise the parser will add
this flags if you tell it to do so.

\section{Finally..}

After you successully retrieved your documents and enjoyed reading
them on your PDA you might want to do this every day. To ease things
up consider putting these calls into a short script file. On OS/2 and
Windows simly create a file called hotsync.cmd (OS/2) or hotsync.bat
(Windows) and place the calls just like you would have entered them on
the command prompt. In case you use dial up conections to the internet
you can even include the calls to build up that connection before the
gathering process starts.\\

Now you should be ready to start using Plucker. The next chapter will
describe the client application on your hand-held, the \emph{Viewer}.
